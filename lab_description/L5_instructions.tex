
\documentclass{report}   

\usepackage{graphicx}
\usepackage[labelformat=empty]{caption}
\usepackage{subcaption}

\usepackage{multirow}
\usepackage{graphics,graphicx} % for pdf, bitmapped graphics files
\usepackage{epsfig} % for postscript graphics files
%\usepackage{mathptmx} % assume new font selection scheme installed
\usepackage{times} % assumes new font selection scheme installed
%\usepackage{amssymb}  % assumes amsmath package installed

\usepackage{amsmath} % assumes amsmath package installed
\usepackage{amsfonts}
%\usepackage{amssymb}  % assumes amsmath package installed
\usepackage{url}

\newcommand{\Rnum}{\mathbb{R}} % Symbol fo the real numbers set
\newcommand{\mat}[1]{\ensuremath{\begin{bmatrix}#1\end{bmatrix}}}	% matrix
\newcommand{\myparagraph}[1]{\paragraph{#1}\mbox{}\\}
\DeclareMathOperator*{\argmin}{\arg\!\min}				% argmin


\begin{document}


\section*{LAB 5: Floating base robot: quasi-static control of stability}

Notes:
\begin{itemize}
	\item You can check the frequency of the controller inspecting the topic \textbf{/hyq/command} with \textbf{rostopic hz /hyq/command}.
	\item It is possible to set the verbosity of the output, by setting the parameter  \textit{verbose} in  ex\_5\_conf.py.
\end{itemize}

1) \textit{Sinusoidal reference:} 
Set a sinusoidal reference for $Z$ direction ($f_z$ = 0.5 $Hz$, $A_z$ = 0.05 $rad$ )
together with another for the pith direction ($f_{\theta}$ = 1 $Hz$, $A_{\theta}$ = 0.1 $rad$ )

\quad
 
\noindent  2) \textit{Projection-based controller:} 
Design a controller for base frame position and orientation with a quasi-static approach.

 \textit{2.1)Virtual Impedance:}
Design a virtual impedance to track  references for the base frame position and orientation. Compute the desired wrench $W^d \in\Rnum^6$ to realize the virtual impedance:

\begin{align}
W^d_{lin} &= K_{lin} (x^d_b - x_b ) + D_{lin} (\dot{x_b}^d - \dot{x})   \\
W^d_{ang} &=  - K_{ang} e_o + D_{ang} (\dot{\omega_b}^d - \omega)
\end{align}

where $e_o \in \Rnum^3$ is the orientation error. 

\textit{2.2) Mapping to torques:}
Map the desired wrench into torques  with a projection-based approach:

\begin{equation}
\tau^d = J_{cj}^T(J_b^T)^{\dagger} W^d
\end{equation}

where:
\begin{equation}
J_b^T = \mat{I_{3\times3} & \dots & I_{3\times3} \\
			[x_{f_1} - x_b]_{\times} & \dots & [x_{f_c} - x_b]_{\times}}
\end{equation}

Check the robot is tracking the sinusoidal reference trajectory of point 1). How can this be improved? 

\quad

\noindent 3) \textit{Control of CoM:}
Modify the controller to control the position of the CoM in place of the base frame.
Which changes are necessary?

\noindent 4) \textit{Gravity compensation:}
Using the controller for the CoM position, add a gravity term $W_g$ to the desired wrench.

\begin{align}
W^d_{lin} &= W_{fb} +  W_g
\end{align}

How does $W_g$ look like? (hint: remember that gravity is applied to CoM)
How is the tracking error in steady state? and in motion?

\quad

\noindent 5) \textit{Feed-forward term:}
Add a feed-forward term the desired wrench. 

\begin{align}
W^d_{lin} &= W_{fb} +  W_g + W_{ffwd}
\end{align}
How would you implement that at the impedance level? (hint: exploit the mass matrix and the tensor of inertia).
How is the tracking improved during motion?

\quad

\noindent6) \textit{Check static stability:}

Design an horizontal trajectory for the CoM (e.g. moving along Y direction) starting from the default configuration q0. 
What happens when the CoM goes out of the polygon?

\noindent8) \textit{quasi-static QP controller:}
Re-implement the previous mapping from wrench to torques by casting as an optimization problem (QP) that optimizes for ground reaction forces $f^d \in \Rnum^k$. 
Enforce unilateral constraints for the leg that are in stance $f_{z,min} = 0$. 

\begin{equation} 
\begin{aligned} 
f^d = & \argmin_{f \in \Rnum^k} (Af-b)^T S (Af-b) + \alpha f^T W f\\
& s.t. \quad \underline{d} < C f < \bar{d},\\
\end{aligned} 
\label{eq:min_prob}
\end{equation}
%
where:
\begin{equation}
C = \mat{C_0 & \dots & 0 \\ \vdots & \ddots & \vdots \\ 0 & \dots & C_c} \quad
\underline{d} = \mat{ \underline{d}_0 \\ \vdots \\ \underline{d}_c} \quad
\bar{d} = \mat{\bar{d}_0 \\ \vdots \\ \bar{d}_c},
\end{equation}
%
with: $A = J_b^T$ and $b = W^d$, $C_i = 	n_i^T$, $\underline{d}_i = 	f_{{min}_i}$, $\bar{d}_i =f_{{max}_i}$. 
Why a regularization (the second term in the cost function with wight $\alpha$) is useful?
Compare with the previous projection-based controller of point 6). 
Does it tolerates higher frequencies on the reference trajectories?
Check the $Z$ component of the ground reaction forces is always positive.


\end{document}