
\documentclass{report}   

\usepackage{graphicx}
\usepackage[labelformat=empty]{caption}
\usepackage{subcaption}

\usepackage{multirow}
\usepackage{graphics,graphicx} % for pdf, bitmapped graphics files
\usepackage{epsfig} % for postscript graphics files
%\usepackage{mathptmx} % assume new font selection scheme installed
\usepackage{times} % assumes new font selection scheme installed
%\usepackage{amssymb}  % assumes amsmath package installed

\usepackage{amsmath} % assumes amsmath package installed
\usepackage{amsfonts}
%\usepackage{amssymb}  % assumes amsmath package installed
\usepackage{url}

\newcommand{\Rnum}{\mathbb{R}} % Symbol fo the real numbers set
\newcommand{\mat}[1]{\ensuremath{\begin{bmatrix}#1\end{bmatrix}}}	% matrix
\newcommand{\myparagraph}[1]{\paragraph{#1}\mbox{}\\}

\begin{document}


\section*{LAB 4: Centroidal floating base  dynamics}

This lab is to acquire confidence to some invariant properties of floating base dynamics. 
The base frame B of a floating base robot is usually placed at the robot waist/trunk because this choice induces a  branch sparsity of the Jacobians. 
However, by defining a different parametrization for the floating base, it is possible to obtain a \textit{block diagonal} joint space inertia matrix.  This is achieved by defining a variable changes for the state of the system using the Center of Mass velocity and the \textit{average} angular velocity.  It can be shown that the mechanical power is invariant with respect to the change of coordinate frames used to represent the floating base. 

1) \textit{Center of mass computation:}
Compute the Center of Mass of the system performing a weighted average of the Center of Masses of all the links. 

2) \textit{Kinetic energy:}
Compute the kinetic energy of the system in two different ways: one by the generalized velocities and system mass matrix, another summing up the contributions from each joint, showing the result is the same. 

3) \textit{Centroidal coordinates transform:} 
Build the transformation matrix ${}_gT_b$ to change coordinates from base to com frame.


\begin{align}
{}_GT_B  = \mat{    {}_GX_B  & I_G^{-1} {}_GX_B^{*} F_B  \\ 0_{n \times 6} &   I_{n \times n} }
\end{align}

where: 

\begin{align}
 {}_GX_B = \mat{ I_{n \times n}  & -[x_{b, com}]_{\times} \\ 0_{3 \times 3} & 	 I_{3 \times 3}}
 	\label{fig:}
\end{align}

check 

4) \textit{Block-diagonal joint space inertia matrix:}
Show the mass matrix is decoupled if you apply coordinate transform to com variables:

\begin{align}
M_G = ({}_GT_B)^{-T} M_B ({}_GT_B)^{-1}
\label{fig:}
\end{align}

5) \textit{Joint gravity vector nullifies in com coordinates:}
\begin{align}
G_G = -M_g \mat{g_w \\ 0_{n+3}} = \mat{0 \\0\\ -mg  \\ 0_{3x1} \\0_{nx1} }
\label{fig:}
\end{align}

\begin{align}
G_G =  ({}_GT_B)^{-T} G_B
\label{fig:}
\end{align}

\end{document}